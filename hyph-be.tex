% Created by: TBD
% License: TBD
%
% Створана адпаведна правілаў, зацверджаных законам
% ад 23 лiпеня 2008 г. No 420-З
% Аб Правiлах беларускай арфаграфii i пунктуацыi
% http://academy.edu.by/files/zak_420-3.pdf
%
% РАЗДЗЕЛ I. АРФАГРАФIЯ
% ГЛАВА 8 ПРАВIЛЫ ПЕРАНОСУ
% § 41. Правiлы пераносу простых, складаных, складанаскарочаных слоў,
%       умоўных графiчных скарачэнняў i iншых знакаў
%
% Based on rules approved by the law No 420-3 dated 23 July 2008
% About Belarusian orthography and punctuation
%
\message{Loading hyphenation patterns for Belarusian}
\patterns{
%
% З аднаго радка на другi слова пераносiцца па складах.
% Words are hyphenated by syllables.
%
% во-ля, тра-ва, за-яц, га-ла-ва, ка-ва-лак,
% стра-ка-ты, пра-ве-рыць, пе-ра-кi-нуць.
%
а1
е1
ё1
і1
о1
у1
ы1
э1
ю1
я1
%
%   2. Калi ў сярэдзiне слова памiж галоснымi маецца спалучэнне зычных, то
% пераносiцца на наступны радок або ўсё гэта спалучэнне, або любая яго частка.
% If there are several consonants between vowels, all of them may be hyphenated
% or any part of it as well.
%
% ся-стра, сяс-тра, сяст-ра
% во-стры, вос-тры, вост-ры
% пту-шка, птуш-ка
% кро-пля, кроп-ля
% ма-ста-цтва, мас-тац-тва, мас-тацт-ва;
% ра-змова, раз-мова
% за-става, зас-тава
% ра-скрыць, рас-крыць, раск-рыць
% бя-скрыўдна, бяс-крыўдна, бяск-рыўдна
% дзя-цi-нства, дзя-цiн-ства, дзя-цiнс-тва, дзя-цiнст-ва
% двац-цаць, два-ццаць
% калос-се, кало-ссе
% сол-лю, со-ллю
% памяц-цю, памя-ццю
% мыц-ца, мы-цца
% паа-бапал, па-абапал
% насен-не, насе-нне
%
б5б
б3в
б3г
б3ґ
б3д
б3ж
б3з
б3к
б3л
б3м
б3н
б3п
б3р
б3с
б3т
б3ф
б3х
б3ц
б3ч
б3ш
в3б
в5в
в3г
в3ґ
в3д
в3ж
в3з
в3к
в3л
в3м
в3н
в3п
в3р
в3с
в3т
в3ф
в3х
в3ц
в3ч
в3ш
г3б
г3в
г5г
г3ґ
г3д
г3ж
г3з
г3к
г3л
г3м
г3н
г3п
г3р
г3с
г3т
г3ф
г3х
г3ц
г3ч
г3ш
ґ3б
ґ3в
ґ3г
ґ5ґ
ґ3д
ґ3ж
ґ3з
ґ3к
ґ3л
ґ3м
ґ3н
ґ3п
ґ3р
ґ3с
ґ3т
ґ3ф
ґ3х
ґ3ц
ґ3ч
ґ3ш
д3б
д3в
д3г
д3ґ
д5д
% дж % special case
% дз % special case
д3к
д3л
д3м
д3н
д3п
д3р
д3с
д3т
д3ф
д3х
д3ц
д3ч
д3ш
ж3б
ж3в
ж3г
ж3ґ
ж3д
ж5ж
ж3з
ж3к
ж3л
ж3м
ж3н
ж3п
ж3р
ж3с
ж3т
ж3ф
ж3х
ж3ц
ж3ч
ж3ш
з3б
з3в
з3г
з3ґ
з3д
з3ж
з5з
з3к
з3л
з3м
з3н
з3п
з3р
з3с
з3т
з3ф
з3х
з3ц
з3ч
з3ш
к3б
к3в
к3г
к3ґ
к3д
к3ж
к3з
к5к
к3л
к3м
к3н
к3п
к3р
к3с
к3т
к3ф
к3х
к3ц
к3ч
к3ш
л3б
л3в
л3г
л3ґ
л3д
л3ж
л3з
л3к
л5л
л3м
л3н
л3п
л3р
л3с
л3т
л3ф
л3х
л3ц
л3ч
л3ш
м3б
м3в
м3г
м3ґ
м3д
м3ж
м3з
м3к
м3л
м5м
м3н
м3п
м3р
м3с
м3т
м3ф
м3х
м3ц
м3ч
м3ш
н3б
н3в
н3г
н3ґ
н3д
н3ж
н3з
н3к
н3л
н3м
н5н
н3п
н3р
н3с
н3т
н3ф
н3х
н3ц
н3ч
н3ш
п3б
п3в
п3г
п3ґ
п3д
п3ж
п3з
п3к
п3л
п3м
п3н
п5п
п3р
п3с
п3т
п3ф
п3х
п3ц
п3ч
п3ш
р3б
р3в
р3г
р3ґ
р3д
р3ж
р3з
р3к
р3л
р3м
р3н
р3п
р5р
р3с
р3т
р3ф
р3х
р3ц
р3ч
р3ш
с3б
с3в
с3г
с3ґ
с3д
с3ж
с3з
с3к
с3л
с3м
с3н
с3п
с3р
с5с
с3т
с3ф
с3х
с3ц
с3ч
с3ш
т3б
т3в
т3г
т3ґ
т3д
т3ж
т3з
т3к
т3л
т3м
т3н
т3п
т3р
т3с
т5т
т3ф
т3х
т3ц
т3ч
т3ш
ф3б
ф3в
ф3г
ф3ґ
ф3д
ф3ж
ф3з
ф3к
ф3л
ф3м
ф3н
ф3п
ф3р
ф3с
ф3т
ф5ф
ф3х
ф3ц
ф3ч
ф3ш
х3б
х3в
х3г
х3ґ
х3д
х3ж
х3з
х3к
х3л
х3м
х3н
х3п
х3р
х3с
х3т
х3ф
х5х
х3ц
х3ч
х3ш
ц3б
ц3в
ц3г
ц3ґ
ц3д
ц3ж
ц3з
ц3к
ц3л
ц3м
ц3н
ц3п
ц3р
ц3с
ц3т
ц3ф
ц3х
ц5ц
ц3ч
ц3ш
ч3б
ч3в
ч3г
ч3ґ
ч3д
ч3ж
ч3з
ч3к
ч3л
ч3м
ч3н
ч3п
ч3р
ч3с
ч3т
ч3ф
ч3х
ч3ц
ч5ч
ч3ш
ш3б
ш3в
ш3г
ш3ґ
ш3д
ш3ж
ш3з
ш3к
ш3л
ш3м
ш3н
ш3п
ш3р
ш3с
ш3т
ш3ф
ш3х
ш3ц
ш3ч
ш5ш
%
% Пры пераносе нельга пакiдаць або пераносiць на наступны радок адну лiтару,
% нават калi яна адпавядае складу.
% A single letter can't be left of hyphenated, even if it forms a syllable.
%
% аса-ка, лi-нiя, ра-дыё, еха-лi, па-коi
%
.а8
8а.
.е8
8е.
.ё8
8ё.
.і8
8і.
.о8
8о.
.у8
8у.
.ы8
8ы.
.э8
8э.
.ю8
8ю.
.я8
8я.
%
% Пры пераносе нельга разбiваць пераносам спалучэннi лiтар дж i дз,
% калi яны абазначаюць адзiн гук [дж], [дз’]:
% Combinations дж and дз shouldn't be split if they form a single sound.
%
% ура-джай, са-джаць, ра-дзi-ма, ха-дзiць
%
% Спалучэннi дж i дз можна разбiваць пераносам, калi д адносiцца да прыстаўкi,
% а з, ж – да кораня:
% These combinations can be split if д belongs to a prefix
% and ж or з to the root of a word.
%
% пад-жары, ад-жаць, пад-земны, ад-значыць
%
д2ж
д2з
% Пры пераносе нельга аддзяляць ад папярэдняй галоснай лiтары й i ў.
% й and ў should not be separated from a preceding vowel.
%
% сой-ка, бой-кi, май-стар, дай-сцi, зай-мацца, праў-да, слоў-нiк, маў-чаць,
% заў-тра, праў-нук
%
% ** й ды ў ужываюцца выключна пасля галосных,
% таму можна забарону пераносу зрабіть агульнай для усіх літар.
% ** since й and ў can appear only after a vowel
% we can use a more general rule.
%
6й1
6ў1
% Пры пераносе нельга  аддзяляць мяккi знак i апостраф ад папярэдняй зычнай.
% ь and ' should not be separated from a preceding consonant.
%
% буль-ба, прось-ба, вазь-му, бур’-ян, сем’-яў, мыш’-як
%
% ** Зычная з мяккім знакам ці апострафам з'яўляюцца часткай папярэдняга складу.
% ** A consonant and following ь or ' are part of a previous syllable.
%
6б'1
6в'1
6г'1
6ґ'1
6д'1
6ж'1
6з'1
6к'1
6л'1
6м'1
6н'1
6п'1
6р'1
6с'1
6т'1
6ф'1
6х'1
6ц'1
6ч'1
6ш'1
6бь1
6вь1
6гь1
6ґь1
% дь % impossible
% жь % impossible
6зь1
6кь1
6ль1
6мь1
6нь1
6пь1
% рь % impossible
6сь1
% ть % impossible
6фь1
6хь1
6ць1
% чь % impossible
% шь % impossible
%
% Зычныя літары не утвараюць склад, таму не пераносяцца,
% калі стаяць у пачатку ці канцы слова.
% Since consonants do not form syllables, it is not allowed to separate
% consonants in the beginning or ending of a word.
%
.б8
8б.
.в8
8в.
.г8
8г.
.ґ8
8ґ.
.д8
8д.
.ж8
8ж.
.з8
8з.
.к8
8к.
.л8
8л.
.м8
8м.
.н8
8н.
.п8
8п.
.р8
8р.
.с8
8с.
.т8
8т.
.ф8
8ф.
.х8
8х.
.ц8
8ц.
.ч8
8ч.
.ш8
8ш.
.й8
8й.
.ў8
8ў.
.бб8
8бб.
.бв8
8бв.
.бг8
8бг.
.бґ8
8бґ.
.бд8
8бд.
.бж8
8бж.
.бз8
8бз.
.бк8
8бк.
.бл8
8бл.
.бм8
8бм.
.бн8
8бн.
.бп8
8бп.
.бр8
8бр.
.бс8
8бс.
.бт8
8бт.
.бф8
8бф.
.бх8
8бх.
.бц8
8бц.
.бч8
8бч.
.бш8
8бш.
.вб8
8вб.
.вв8
8вв.
.вг8
8вг.
.вґ8
8вґ.
.вд8
8вд.
.вж8
8вж.
.вз8
8вз.
.вк8
8вк.
.вл8
8вл.
.вм8
8вм.
.вн8
8вн.
.вп8
8вп.
.вр8
8вр.
.вс8
8вс.
.вт8
8вт.
.вф8
8вф.
.вх8
8вх.
.вц8
8вц.
.вч8
8вч.
.вш8
8вш.
.гб8
8гб.
.гв8
8гв.
.гг8
8гг.
.гґ8
8гґ.
.гд8
8гд.
.гж8
8гж.
.гз8
8гз.
.гк8
8гк.
.гл8
8гл.
.гм8
8гм.
.гн8
8гн.
.гп8
8гп.
.гр8
8гр.
.гс8
8гс.
.гт8
8гт.
.гф8
8гф.
.гх8
8гх.
.гц8
8гц.
.гч8
8гч.
.гш8
8гш.
.ґб8
8ґб.
.ґв8
8ґв.
.ґг8
8ґг.
.ґґ8
8ґґ.
.ґд8
8ґд.
.ґж8
8ґж.
.ґз8
8ґз.
.ґк8
8ґк.
.ґл8
8ґл.
.ґм8
8ґм.
.ґн8
8ґн.
.ґп8
8ґп.
.ґр8
8ґр.
.ґс8
8ґс.
.ґт8
8ґт.
.ґф8
8ґф.
.ґх8
8ґх.
.ґц8
8ґц.
.ґч8
8ґч.
.ґш8
8ґш.
.дб8
8дб.
.дв8
8дв.
.дг8
8дг.
.дґ8
8дґ.
.дд8
8дд.
.дж8
8дж.
.дз8
8дз.
.дк8
8дк.
.дл8
8дл.
.дм8
8дм.
.дн8
8дн.
.дп8
8дп.
.др8
8др.
.дс8
8дс.
.дт8
8дт.
.дф8
8дф.
.дх8
8дх.
.дц8
8дц.
.дч8
8дч.
.дш8
8дш.
.жб8
8жб.
.жв8
8жв.
.жг8
8жг.
.жґ8
8жґ.
.жд8
8жд.
.жж8
8жж.
.жз8
8жз.
.жк8
8жк.
.жл8
8жл.
.жм8
8жм.
.жн8
8жн.
.жп8
8жп.
.жр8
8жр.
.жс8
8жс.
.жт8
8жт.
.жф8
8жф.
.жх8
8жх.
.жц8
8жц.
.жч8
8жч.
.жш8
8жш.
.зб8
8зб.
.зв8
8зв.
.зг8
8зг.
.зґ8
8зґ.
.зд8
8зд.
.зж8
8зж.
.зз8
8зз.
.зк8
8зк.
.зл8
8зл.
.зм8
8зм.
.зн8
8зн.
.зп8
8зп.
.зр8
8зр.
.зс8
8зс.
.зт8
8зт.
.зф8
8зф.
.зх8
8зх.
.зц8
8зц.
.зч8
8зч.
.зш8
8зш.
.кб8
8кб.
.кв8
8кв.
.кг8
8кг.
.кґ8
8кґ.
.кд8
8кд.
.кж8
8кж.
.кз8
8кз.
.кк8
8кк.
.кл8
8кл.
.км8
8км.
.кн8
8кн.
.кп8
8кп.
.кр8
8кр.
.кс8
8кс.
.кт8
8кт.
.кф8
8кф.
.кх8
8кх.
.кц8
8кц.
.кч8
8кч.
.кш8
8кш.
.лб8
8лб.
.лв8
8лв.
.лг8
8лг.
.лґ8
8лґ.
.лд8
8лд.
.лж8
8лж.
.лз8
8лз.
.лк8
8лк.
.лл8
8лл.
.лм8
8лм.
.лн8
8лн.
.лп8
8лп.
.лр8
8лр.
.лс8
8лс.
.лт8
8лт.
.лф8
8лф.
.лх8
8лх.
.лц8
8лц.
.лч8
8лч.
.лш8
8лш.
.мб8
8мб.
.мв8
8мв.
.мг8
8мг.
.мґ8
8мґ.
.мд8
8мд.
.мж8
8мж.
.мз8
8мз.
.мк8
8мк.
.мл8
8мл.
.мм8
8мм.
.мн8
8мн.
.мп8
8мп.
.мр8
8мр.
.мс8
8мс.
.мт8
8мт.
.мф8
8мф.
.мх8
8мх.
.мц8
8мц.
.мч8
8мч.
.мш8
8мш.
.нб8
8нб.
.нв8
8нв.
.нг8
8нг.
.нґ8
8нґ.
.нд8
8нд.
.нж8
8нж.
.нз8
8нз.
.нк8
8нк.
.нл8
8нл.
.нм8
8нм.
.нн8
8нн.
.нп8
8нп.
.нр8
8нр.
.нс8
8нс.
.нт8
8нт.
.нф8
8нф.
.нх8
8нх.
.нц8
8нц.
.нч8
8нч.
.нш8
8нш.
.пб8
8пб.
.пв8
8пв.
.пг8
8пг.
.пґ8
8пґ.
.пд8
8пд.
.пж8
8пж.
.пз8
8пз.
.пк8
8пк.
.пл8
8пл.
.пм8
8пм.
.пн8
8пн.
.пп8
8пп.
.пр8
8пр.
.пс8
8пс.
.пт8
8пт.
.пф8
8пф.
.пх8
8пх.
.пц8
8пц.
.пч8
8пч.
.пш8
8пш.
.рб8
8рб.
.рв8
8рв.
.рг8
8рг.
.рґ8
8рґ.
.рд8
8рд.
.рж8
8рж.
.рз8
8рз.
.рк8
8рк.
.рл8
8рл.
.рм8
8рм.
.рн8
8рн.
.рп8
8рп.
.рр8
8рр.
.рс8
8рс.
.рт8
8рт.
.рф8
8рф.
.рх8
8рх.
.рц8
8рц.
.рч8
8рч.
.рш8
8рш.
.сб8
8сб.
.св8
8св.
.сг8
8сг.
.сґ8
8сґ.
.сд8
8сд.
.сж8
8сж.
.сз8
8сз.
.ск8
8ск.
.сл8
8сл.
.см8
8см.
.сн8
8сн.
.сп8
8сп.
.ср8
8ср.
.сс8
8сс.
.ст8
8ст.
.сф8
8сф.
.сх8
8сх.
.сц8
8сц.
.сч8
8сч.
.сш8
8сш.
.тб8
8тб.
.тв8
8тв.
.тг8
8тг.
.тґ8
8тґ.
.тд8
8тд.
.тж8
8тж.
.тз8
8тз.
.тк8
8тк.
.тл8
8тл.
.тм8
8тм.
.тн8
8тн.
.тп8
8тп.
.тр8
8тр.
.тс8
8тс.
.тт8
8тт.
.тф8
8тф.
.тх8
8тх.
.тц8
8тц.
.тч8
8тч.
.тш8
8тш.
.фб8
8фб.
.фв8
8фв.
.фг8
8фг.
.фґ8
8фґ.
.фд8
8фд.
.фж8
8фж.
.фз8
8фз.
.фк8
8фк.
.фл8
8фл.
.фм8
8фм.
.фн8
8фн.
.фп8
8фп.
.фр8
8фр.
.фс8
8фс.
.фт8
8фт.
.фф8
8фф.
.фх8
8фх.
.фц8
8фц.
.фч8
8фч.
.фш8
8фш.
.хб8
8хб.
.хв8
8хв.
.хг8
8хг.
.хґ8
8хґ.
.хд8
8хд.
.хж8
8хж.
.хз8
8хз.
.хк8
8хк.
.хл8
8хл.
.хм8
8хм.
.хн8
8хн.
.хп8
8хп.
.хр8
8хр.
.хс8
8хс.
.хт8
8хт.
.хф8
8хф.
.хх8
8хх.
.хц8
8хц.
.хч8
8хч.
.хш8
8хш.
.цб8
8цб.
.цв8
8цв.
.цг8
8цг.
.цґ8
8цґ.
.цд8
8цд.
.цж8
8цж.
.цз8
8цз.
.цк8
8цк.
.цл8
8цл.
.цм8
8цм.
.цн8
8цн.
.цп8
8цп.
.цр8
8цр.
.цс8
8цс.
.цт8
8цт.
.цф8
8цф.
.цх8
8цх.
.цц8
8цц.
.цч8
8цч.
.цш8
8цш.
.чб8
8чб.
.чв8
8чв.
.чг8
8чг.
.чґ8
8чґ.
.чд8
8чд.
.чж8
8чж.
.чз8
8чз.
.чк8
8чк.
.чл8
8чл.
.чм8
8чм.
.чн8
8чн.
.чп8
8чп.
.чр8
8чр.
.чс8
8чс.
.чт8
8чт.
.чф8
8чф.
.чх8
8чх.
.чц8
8чц.
.чч8
8чч.
.чш8
8чш.
.шб8
8шб.
.шв8
8шв.
.шг8
8шг.
.шґ8
8шґ.
.шд8
8шд.
.шж8
8шж.
.шз8
8шз.
.шк8
8шк.
.шл8
8шл.
.шм8
8шм.
.шн8
8шн.
.шп8
8шп.
.шр8
8шр.
.шс8
8шс.
.шт8
8шт.
.шф8
8шф.
.шх8
8шх.
.шц8
8шц.
.шч8
8шч.
.шш8
8шш.
%
% Стылістычна лепш не аддзяляць прыстаўку не- ад слова. Інакш, гэта можа
% паўплываць на разуменне тексту.
%
% It is better not to separate prefix не- from a word.
% Alternatively this can affect understanding.
%
.не8
.ня8
%
% Перанос пасля некаторых прыставак толькі для слоў пачынаючыхся з зычных.
% Калі пасля ідзе галосная, вельмі верагодна што гэта не прыстаўка,
% а частка кораня.
%
% Hyphenation after several prefixes only for following consonants.
% If following letter is a vowel, it's very likely that the prefix
% is actually a part of the root.
%
% ад-
.ад3б
.ад3в
.ад3г
.ад3ґ
.ад3д
.ад3ж
.ад3з
.ад3к
.ад3л
.ад3м
.ад3н
.ад3п
.ад3р
.ад3с
.ад3т
.ад3ф
.ад3х
.ад3ц
.ад3ч
.ад3ш
% над-
.на2д3б
.на2д3в
.на2д3г
.на2д3ґ
.на2д3д
.на2д3ж
.на2д3з
.на2д3к
.на2д3л
.на2д3м
.на2д3н
.на2д3п
.на2д3р
.на2д3с
.на2д3т
.на2д3ф
.на2д3х
.на2д3ц
.на2д3ч
.на2д3ш
% пад-
.па2д3б
.па2д3в
.па2д3г
.па2д3ґ
.па2д3д
.па2д3ж
.па2д3з
.па2д3к
.па2д3л
.па2д3м
.па2д3н
.па2д3п
.па2д3р
.па2д3с
.па2д3т
.па2д3ф
.па2д3х
.па2д3ц
.па2д3ч
.па2д3ш
% перад-
.пера2д3б
.пера2д3в
.пера2д3г
.пера2д3ґ
.пера2д3д
.пера2д3ж
.пера2д3з
.пера2д3к
.пера2д3л
.пера2д3м
.пера2д3н
.пера2д3п
.пера2д3р
.пера2д3с
.пера2д3т
.пера2д3ф
.пера2д3х
.пера2д3ц
.пера2д3ч
.пера2д3ш
% аб-
.аб3б
.аб3в
.аб3г
.аб3ґ
.аб3д
.аб3ж
.аб3з
.аб3к
.аб3л
.аб3м
.аб3н
.аб3п
.аб3р
.аб3с
.аб3т
.аб3ф
.аб3х
.аб3ц
.аб3ч
.аб3ш
% раз-
.ра2з3б
.ра2з3в
.ра2з3г
.ра2з3ґ
.ра2з3д
.ра2з3ж
.ра2з3з
.ра2з3к
.ра2з3л
.ра2з3м
.ра2з3н
.ра2з3п
.ра2з3р
.ра2з3с
.ра2з3т
.ра2з3ф
.ра2з3х
.ра2з3ц
.ра2з3ч
.ра2з3ш
% уз-
.уз3б
.уз3в
.уз3г
.уз3ґ
.уз3д
.уз3ж
.уз3з
.уз3к
.уз3л
.уз3м
.уз3н
.уз3п
.уз3р
.уз3с
.уз3т
.уз3ф
.уз3х
.уз3ц
.уз3ч
.уз3ш
% праз-
.пра2з3б
.пра2з3в
.пра2з3г
.пра2з3ґ
.пра2з3д
.пра2з3ж
.пра2з3з
.пра2з3к
.пра2з3л
.пра2з3м
.пра2з3н
.пра2з3п
.пра2з3р
.пра2з3с
.пра2з3т
.пра2з3ф
.пра2з3х
.пра2з3ц
.пра2з3ч
.пра2з3ш
% ус-
.ус3б
.ус3в
.ус3г
.ус3ґ
.ус3д
.ус3ж
.ус3з
.ус3к
.ус3л
.ус3м
.ус3н
.ус3п
.ус3р
.ус3с
.ус3т
.ус3ф
.ус3х
.ус3ц
.ус3ч
.ус3ш
% рас-
.ра2с3б
.ра2с3в
.ра2с3г
.ра2с3ґ
.ра2с3д
.ра2с3ж
.ра2с3з
.ра2с3к
.ра2с3л
.ра2с3м
.ра2с3н
.ра2с3п
.ра2с3р
.ра2с3с
.ра2с3т
.ра2с3ф
.ра2с3х
.ра2с3ц
.ра2с3ч
.ра2с3ш
% роз-
.ро2з3б
.ро2з3в
.ро2з3г
.ро2з3ґ
.ро2з3д
.ро2з3ж
.ро2з3з
.ро2з3к
.ро2з3л
.ро2з3м
.ро2з3н
.ро2з3п
.ро2з3р
.ро2з3с
.ро2з3т
.ро2з3ф
.ро2з3х
.ро2з3ц
.ро2з3ч
.ро2з3ш
% бяз-
.бя2з3б
.бя2з3в
.бя2з3г
.бя2з3ґ
.бя2з3д
.бя2з3ж
.бя2з3з
.бя2з3к
.бя2з3л
.бя2з3м
.бя2з3н
.бя2з3п
.бя2з3р
.бя2з3с
.бя2з3т
.бя2з3ф
.бя2з3х
.бя2з3ц
.бя2з3ч
.бя2з3ш
% без-
.бе2з3б
.бе2з3в
.бе2з3г
.бе2з3ґ
.бе2з3д
.бе2з3ж
.бе2з3з
.бе2з3к
.бе2з3л
.бе2з3м
.бе2з3н
.бе2з3п
.бе2з3р
.бе2з3с
.бе2з3т
.бе2з3ф
.бе2з3х
.бе2з3ц
.бе2з3ч
.бе2з3ш
% бяс-
.бя2с3б
.бя2с3в
.бя2с3г
.бя2с3ґ
.бя2с3д
.бя2с3ж
.бя2с3з
.бя2с3к
.бя2с3л
.бя2с3м
.бя2с3н
.бя2с3п
.бя2с3р
.бя2с3с
.бя2с3т
.бя2с3ф
.бя2с3х
.бя2с3ц
.бя2с3ч
.бя2с3ш
% бес-
.бе2с3б
.бе2с3в
.бе2с3г
.бе2с3ґ
.бе2с3д
.бе2с3ж
.бе2с3з
.бе2с3к
.бе2с3л
.бе2с3м
.бе2с3н
.бе2с3п
.бе2с3р
.бе2с3с
.бе2с3т
.бе2с3ф
.бе2с3х
.бе2с3ц
.бе2с3ч
.бе2с3ш
% рос-
.ро2с3б
.ро2с3в
.ро2с3г
.ро2с3ґ
.ро2с3д
.ро2с3ж
.ро2с3з
.ро2с3к
.ро2с3л
.ро2с3м
.ро2с3н
.ро2с3п
.ро2с3р
.ро2с3с
.ро2с3т
.ро2с3ф
.ро2с3х
.ро2с3ц
.ро2с3ч
.ро2с3ш
% цераз-
.цера2з3б
.цера2з3в
.цера2з3г
.цера2з3ґ
.цера2з3д
.цера2з3ж
.цера2з3з
.цера2з3к
.цера2з3л
.цера2з3м
.цера2з3н
.цера2з3п
.цера2з3р
.цера2з3с
.цера2з3т
.цера2з3ф
.цера2з3х
.цера2з3ц
.цера2з3ч
.цера2з3ш
% церас-
.цера2с3б
.цера2с3в
.цера2с3г
.цера2с3ґ
.цера2с3д
.цера2с3ж
.цера2с3з
.цера2с3к
.цера2с3л
.цера2с3м
.цера2с3н
.цера2с3п
.цера2с3р
.цера2с3с
.цера2с3т
.цера2с3ф
.цера2с3х
.цера2с3ц
.цера2с3ч
.цера2с3ш
% Перанос у словах, якія ўяўляюць сабой
% выключэнні з вышэй  апісаных правілаў
.ад8зін
тэ8мбр.
8льш.
8сць.
8дзь.
.дву8х3
.тро8х3
слова7ў8твар
віда1з8мян
віда1з8мен
за1п8люшч
.па5д8зял
.па5д8зел
вё8рст
раз5г8ляд
раз5г8лед
зло1ў8жыв
.па5д8зяк
.вы1к8люч
.шма8т1
крова3ў8твар
за3ц8вярдз
.па3г8лядз
на5д8вор
}

